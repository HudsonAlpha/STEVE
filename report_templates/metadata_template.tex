\section{Sample metadata}
\subsection{General Sample Info}
This section contains information regarding where samples were acquired from.  This corresponds to the ``Sample'' label in Table \ref{tab:sample_metadata}.

\begin{enumerate}
    \item NA12878 - female of European ancestry; purchased through \url{https://www.coriell.org/0/Sections/Search/Sample_Detail.aspx?Ref=NA12878&Product=DNA}
    \item HG002-HG004 - son and parents of Eastern Europe Ashkenazi Jewish ancestry; purchased through \url{https://www-s.nist.gov/srmors/view_detail.cfm?srm=8392}
    \item HG005 - male of Chinese ancestry; purchased through \url{https://www-s.nist.gov/srmors/view_detail.cfm?srm=8393}
\end{enumerate}

\subsection{Training Samples}
This section contains information regarding the specific samples used for analysis.  This data is automatically pulled from a sample JSON file containing sample names, sample types (i.e. which GIAB sample), and how the sample was prepared.  Table \ref{tab:sample_metadata} contains the list of metadata as pulled from the JSON.

% NOTE TO USER: if samples are added, make sure the sample JSON is updated or it won't get caught

\begin{longtable}{|l|r|r|r|}
    \hline
    {{ FORMAT.HEADER_COLOR }}\textbf{Library}
    
        &{{ '\\textbf{'+FORMAT.METADATA.get(fk, {}).get('label', fk)+'}' }}
    
    \\ \hline
    \endhead
    
        {{ sample.replace('_', '\\_') }}
        
            &{{ FORMAT.METADATA.get(fk, {}).get('format', FORMAT.METADATA['default']['format']).format(METADATA.get(sample, {}).get(fk, 'NOT IN METADATA')) }}
        
        \\ \hline
    
    \caption{This table contains metadata regarding each sequenced sample.  The GIAB sample label and prep type are currently the two pieces of tracked metadata regarding each sample.}
    \label{tab:sample_metadata}
\end{longtable}